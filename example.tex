\documentclass[12pt,a4paper]{article}
\usepackage{styles/preamble}

% =================================================
% НАЧАЛО ДОКУМЕНТА
% =================================================

\begin{document}

\import{titlepage/}{title_old} % Титульный лист
\import{titlepage/}{title_new} % Титульный лист

\tableofcontents % Так выглядит вставка с содержанием

\section{Самая главная секция}

На первых двух страницах приведены два варианта титульника: старый и новый. Далее приведён возможный вариант стил


\subsection{Шрифт}

В качестве шрифтов используются \href{https://fonts.google.com/specimen/Raleway}{Raleway} и \href{https://fonts.google.com/specimen/Anonymous+Pro}{AnonymousPro}.

\begin{multicols}{2}
	\begin{itemize}
		\item Raleway, regular.
		\item \textit{Raleway, italic.}
		\item \textbf{Raleway, bold.}
		\item \textbf{\textit{Raleway, bold italic.}}
		\item \texttt{AnonymousPro, mono.}
		\item \textit{\texttt{AnonymousPro, mono italic.}}
		\item \textbf{\texttt{AnonymousPro, mono bold.}}
		\item \textbf{\texttt{AnonymousPro, mono bold italic.}}
	\end{itemize}
\end{multicols}


\subsection{Математические формулы}

\marginpar{\textcolor{olive}{\textbf{TODO} МБ есть формула поинтереснее?}}

В качестве примера приведена формула интегрирования по частям для определённого интеграла \eqref{simple_equation}.

\begin{equation} \label{simple_equation}
\INT_a^b u dv = uv \Big|_a^b - \INT_a^b v du
\end{equation}


\subsection{Листинг}

В качестве примера приведена простая программа на языке Python (Листинг~\ref{lst:simple_code}).

\lst{python}{code/main.py}
\captionof{listing}{Простая программа на Python}
\label{lst:simple_code}

% Принудительно делаем новую страницу для красоты.
% Вообще, так делать не стоит, т.к. если вдруг что-то поменяете сверху,
% то может смотреться некрасиво
\newpage

%-------------------------------------------------------------------------------
\subsection{Алгоритм}

% Перевод данных об алгоритмах
\renewcommand{\listalgorithmname}{Список алгоритмов}
\floatname{algorithm}{Алгоритм}

% Перевод команд псевдокода
\algrenewcommand\algorithmicwhile{\textbf{До тех пор пока}}
\algrenewcommand\algorithmicdo{\textbf{выполнять}}
\algrenewcommand\algorithmicrepeat{\textbf{Повторять}}
\algrenewcommand\algorithmicuntil{\textbf{Пока выполняется}}
\algrenewcommand\algorithmicend{\textbf{Конец}}
\algrenewcommand\algorithmicif{\textbf{Если}}
\algrenewcommand\algorithmicelse{\textbf{иначе}}
\algrenewcommand\algorithmicthen{\textbf{тогда}}
\algrenewcommand\algorithmicfor{\textbf{Цикл}}
\algrenewcommand\algorithmicforall{\textbf{Выполнить для всех}}
\algrenewcommand\algorithmicfunction{\textbf{Функция}}
\algrenewcommand\algorithmicprocedure{\textbf{Процедура}}
\algrenewcommand\algorithmicloop{\textbf{Зациклить}}
\algrenewcommand\algorithmicrequire{\textbf{Условия:}}
\algrenewcommand\algorithmicensure{\textbf{Обеспечивающие условия:}}
\algrenewcommand\algorithmicreturn{\textbf{Возвратить}}
\algrenewtext{EndWhile}{\textbf{Конец цикла}}
\algrenewtext{EndLoop}{\textbf{Конец зацикливания}}
\algrenewtext{EndFor}{\textbf{Конец цикла}}
\algrenewtext{EndFunction}{\textbf{Конец функции}}
\algrenewtext{EndProcedure}{\textbf{Конец процедуры}}
\algrenewtext{EndIf}{\textbf{Конец условия}}
\algrenewtext{EndFor}{\textbf{Конец цикла}}
\algrenewtext{BeginAlgorithm}{\textbf{Начало алгоритма}}
\algrenewtext{EndAlgorithm}{\textbf{Конец алгоритма}}
\algrenewtext{BeginBlock}{\textbf{Начало блока: }}
\algrenewtext{EndBlock}{\textbf{Конец блока}}
\algrenewtext{ElsIf}{\textbf{иначе если }}

\begin{algorithm}
    \caption{Пример алгоритма}\label{alg:Example1}
    \begin{algorithmic}[1]
        \State $ X = 45 $ \Comment{Пример комментария}
        \For{\textbf{от} i=0 \textbf{до} 5}
            \State $ X = X - 2 $
            \State \Call{find}{X}

            \While{$ Y_2 < 5 $} \Comment{Пояснение к циклу}
                \If{$ quality \ge 9 $}
                \State $ a \gets perfect $
                \ElsIf{$ quality \ge 7 $}
                \State $ a \gets good $
                \ElsIf{$ quality \ge 5 $}
                \State $ a \gets medium $
                \ElsIf{$ quality \ge 3 $}
                \State $ a \gets bad $
                \Else
                \State $ a \gets unusable $
                \EndIf
            \EndWhile

            \State \Return $ X $
            \BeginBlock Прибавление 2:
                \State $ X += 2 $
            \EndBlock
        \EndFor
    \end{algorithmic}
\end{algorithm}

\begin{algorithm}
    \caption{Пример алгоритма для оглавления алгоритмов}\label{alg:Example2}
    \begin{algorithmic}[1]
        \Require $ x \ge 5 $ \Comment{Пояснение условия}
        \Statex
        \While{$ x > -5 $}
        \State $ x \gets x - 1 $
        \EndWhile
    \end{algorithmic}
\end{algorithm}

\listofalgorithms
%-------------------------------------------------------------------------------


\subsection{Картинка}

Можно вставлять картинки. Я так и сделал (Рис.~\ref{pic:luthadel}).
\begin{figure}[!h]
	\centering
	\includegraphics[width=1.0\textwidth]{pic/ricky-ho-mistborn-luthadel-city-rickyho}
	\caption{Mistborn: Luthadel at night by Ricky Ho}
	\label{pic:luthadel}
\end{figure}


\subsection{Красивая нумерация}

Можно красиво нумировать списки:

\begin{enumerate}[(a)]
    \item Вот раз.
    \item А вот два, причём:
    \begin{enumerate}[1 $\to$]
    	\item Можно использовать любые символы для нумерации.
    	\item Даже математические!
    \end{enumerate}
\end{enumerate}


\subsection*{Секция, которая не отобрзится в содержании и без цифры!}

\textcolor{gray}{\xout{этого текста нет на картах!}}


\subsection{Таблички}

Таблицы можно делать с помощью пакета \texttt{csvsimple}\footnote{Для несложных таблиц.} или \texttt{pgfplotstable}\footnote{Для таблиц с большим количеством разнородной информации.}. А можно вставлять напрямую:

\begin{table}[h!]
	\centering
	\begin{tabular}{|c|l|l|} \hline
		& \textbf{Имя} & \textbf{Фамилия} \\ \hline
		1 & Гаррье & Дюбуа \\ \hline
		2 & Ким & Кицураги \\ \hline
	\end{tabular}
	\caption{Главные герои Disco Elysium.}
	\label{table:disco_main_characters}
\end{table}

\end{document}